Magnetic materials are used in many applications found in science and engineering, including electric motors, loudspeakers, and magnetic data storage. Each material has a given magnetic relative permeability \(\mu_r\), which describes how the magnetisation of the material changes when magnetic fields are present. Most permanent magnet materials have a relative permeability slightly larger than unity, but are often modelled with \(\mu_r = 1\), ignoring the effect of permeability. This significantly decreases the complexity of modelling, but introduces errors as the magnitude of permeability increases. Some materials such as iron have a high permeability, and are often modelled by assuming the permeability is infinite, also leading to modelling errors. To reduce these modelling errors, non-unity finite permeability must be considered, but this is difficult to model and analyse, often requiring the use of finite element simulations or iterative solvers.

Several researchers have attempted to analytically model magnetic permeability with varying levels of success. \textcite{Kremers2013} modelled a permanent magnet with non-unity permeability by modifying the magnetisation strength of the magnet based on the permeability. This approach gives extremely fast results for the magnetic field produced by a single magnet with low permeability. However, it does not consider the effect of external fields or magnets, and is only valid for small and constant permeabilities, limiting accuracy. \textcite{Dam2016} extended upon this by calculating the interaction force in two of the three Cartesian directions as one magnet is rotated with non-unity relative permeability. However, the third force component was not derived, and the methodology is only valid for cuboidal magnets. \textcite{Casteren2014} published a more general iterative methodology applicable to any magnet geometry, which was more accurate at the cost of considerably longer calculation times. Their approach assumed constant permeability and required recalculating the magnetic field for each iteration, but could incorporate external fields and arbitrarily large permeabilities. In a recent study, \textcite{Zhang2021} performed a similar analysis by subdividing cuboidal permanent magnets into a large number of smaller cuboidal magnets, with the magnetisation of each augmented by the effects of permeability. The force equations from \textcite{Akoun1984}, which assume unity relative permeability, were implemented on each small cuboid, giving accurate force calculations which incorporate the effect of permeability. \textcite{Forbes2021} also subdivided magnetic materials into smaller volumes of material with similar outcomes. Provided the permeability of each segment remains uniform across its volume, the permeability within each sub-volume was able to vary, allowing the modelling of nonlinear magnetic materials.

\textcite[Sec~2.6]{Harrington1993} presented an interesting methodology in electrostatics, wherein the polarisability of a dielectric body is calculated under the assumption of non-unity relative electric permittivity. This is analogous to finding the magnetisation of a magnetic body with non-unity relative permeability, but in the electrostatic domain rather than the magnetostatic. Their method is based on solving a matrix equation based on electric surface charges and the permittivity of the material, allowing a solution to be found. However, their method uses approximations of the integral equations for the field, leading to small but non-trivial errors in the solution.

The current paper introduces a new methodology for calculating magnetic fields, forces, and torques due to magnetic materials with non-unity permeabilities in a single step, thus avoiding iteration. It is similar in concept to the aforementioned matrix equation given by \textcite[Sec~2.6]{Harrington1993}, but with several advantages. The method presented in this paper uses the exact solution to the field equations, leading to high accuracy. In addition, a constraint is applied to the system to ensure consistency with Gauss' Law for Magnetism, further increasing accuracy. Furthermore, systems with arbitrarily many magnets and systems with external field sources may be analysed with this method. Finally, the methodology presented in this paper includes the evaluation of forces and torques on magnetic bodies through numeric integration, allowing analysis of quasi-static magnetic systems.

This methodology is extremely fast and gives accurate results for any magnet shape due to the use of a triangular surface mesh. The methodology begins by calculating surface charge densities in Section \ref{sec:p4magneticChargeDensity}, before using these results to calculate magnetic fields (Section \ref{sec:p4magneticField}), as well as forces and torques (Section \ref{sec:p4forceAndTorque}). Verification is performed on this methodology using several magnetic configurations in Section \ref{sec:p4verification}. Finally, computational considerations are detailed in Section \ref{sec:p4computationalPerformance} before the paper is concluded.