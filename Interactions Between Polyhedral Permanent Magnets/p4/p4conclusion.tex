This paper has outlined a new algorithm for modelling magnetic materials with nonunity relative permeability. Assuming constant uniform magnetisation and constant permeability, the algorithm uses a one-step matrix inversion to solve for the surface charge density, magnetic field, force, and torque. In contrast to other methods available in literature, this method calculates magnetic field information only once, leading to a considerable increase in calculation speed. In addition, using a triangular surface mesh gives this methodology high versatility, allowing the modelling of any polyhedral magnet, and approximate modelling of any general magnet geometry.

The algorithm was verified against literature and finite element simulations using several magnetic configurations. A simple magnetic configuration was defined with two cube magnets in repulsion with varying permeabilities, with the force between the magnets calculated. The force results showed strong agreement with finite element simulations. In addition, it was shown that the permeability has a significant effect on the force between the magnets, and as such, care must be taken when assuming a relative permeability of unity. A similar magnetic configuration was used to verify the torque, where the magnets maintain a constant distance between their centres, but one magnet is rotated. The torque results were in agreement with finite element simulations, and it was again shown that permeability greatly affects torque. Furthermore, the magnetic field was calculated on the plane between the magnet centres with \(\theta =\ \)\ang{90}. Finite element simulations showed strong agreement, with a negligible error.

This methodology has allowed fast and accurate results for any magnet geometry and configuration. This can be used for fast optimisation of electromagnetic designs, giving more accurate results than analytic methods which assume unity relative permeability, and faster results than finite element simulations.