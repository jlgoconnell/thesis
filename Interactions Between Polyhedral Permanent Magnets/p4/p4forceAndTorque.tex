In addition to the magnetic field, the force and torque on each magnet can be approximated using the surface charge densities. According to the magnetic charge model described by \textcite{Furlani2001} (pp. 136,142), the force and torque on a magnetic body are given respectively by
\begin{equation}
	\mathbf{F} = \iiint_V \rho \mathbf{B}_\text{ext} \ dv + \oiint_S \sigma \mathbf{B}_\text{ext} \ ds
\end{equation}
and
\begin{equation}
	\mathbf{T} = \iiint_V \rho \left( \mathbf{r} \times \mathbf{B}_\text{ext} \right) \ dv + \oiint_S \sigma \left( \mathbf{r} \times \mathbf{B}_\text{ext} \right) \ ds \text{,}
\end{equation}
where \(\rho\) and \(\sigma\) are the volume and surface charge densities of the magnetic body respectively, \(\mathbf{B}_\text{ext}\) is the external magnetic field (including the field generated by other magnetic bodies in the system), and \(\mathbf{r}\) is the vector from the point about which the torque is calculated.

Under the assumptions of constant uniform remanence magnetisation and constant permeability, the volume integrals disappear, leaving only the surface integrals. These equations can be applied to the surface mesh defined earlier, giving equations for the approximate force and torque on magnet \(m\).

To achieve this, the vector of surface charge densities \(\bm{\sigma}\) is split into a vector of charge densities of the surface elements of magnet \(m\), \({}_m\bm{\sigma}\), and all other surface charge densities, \({}_{\overline{m}}\bm{\sigma}\). Similarly, the vector from the centre of each element of magnet \(m\) to the point about which the torque is to be calculated is defined as \({}_m\mathbf{r}\). Submatrices \(\hat{B}^*_x\), \(\hat{B}^*_y\), and \(\hat{B}^*_z\) are constructed by taking the rows corresponding to elements belonging to magnet \(m\) and the columns corresponding to elements not belonging to magnet \(m\) of the matrices \(\hat{B}_x\), \(\hat{B}_y\), and \(\hat{B}_z\) respectively. \({}_m\mathbf{B}_{\text{ext,}x}\), \({}_m\mathbf{B}_{\text{ext,}y}\), and \({}_m\mathbf{B}_{\text{ext,}z}\) are defined as the magnetic field at the centre of each surface element of magnet \(m\) produced by external sources such as coils. In the following equations, the operator \(\odot\) is defined as the Hadamard (element-wise) product of two vectors. If the total field on each element produced by all other magnets and the external field is defined as
\begin{align*}
    _m\mathbf{B}_x &= \hat{B}^*_x\ {}_{\overline{m}}\bm{\sigma} + {}_m\mathbf{B}_{\text{ext,}x} \text{,} \\
    _m\mathbf{B}_y &= \hat{B}^*_y\ {}_{\overline{m}}\bm{\sigma} + {}_m\mathbf{B}_{\text{ext,}y} \text{,} \\
    _m\mathbf{B}_z &= \hat{B}^*_z\ {}_{\overline{m}}\bm{\sigma} + {}_m\mathbf{B}_{\text{ext,}z} \text{,}
\end{align*}
the force and torque on magnet \(m\) can be approximated as
\begin{align}
	_mF_x &\approx \left( {}_m\bm{\sigma} \odot {}_m\mathbf{a} \right)^\mathsf{T} {}_m\mathbf{B}_x \text{,} \\
	_mF_y &\approx \left( {}_m\bm{\sigma} \odot {}_m\mathbf{a} \right)^\mathsf{T} {}_m\mathbf{B}_y \text{,} \\
	_mF_z &\approx \left( {}_m\bm{\sigma} \odot {}_m\mathbf{a} \right)^\mathsf{T} {}_m\mathbf{B}_z \text{,}
\end{align}
and
\begin{align}
	_mT_x &\approx \left( {}_m\bm{\sigma} \odot {}_m\mathbf{a} \right)^\mathsf{T} \left( {}_m\mathbf{r}_y \odot {}_m\mathbf{B}_z - {}_m\mathbf{r}_z \odot {}_m\mathbf{B}_y \right) \text{,} \\
	_mT_y &\approx \left( {}_m\bm{\sigma} \odot {}_m\mathbf{a} \right)^\mathsf{T} \left( {}_m\mathbf{r}_z \odot {}_m\mathbf{B}_x - {}_m\mathbf{r}_x \odot {}_m\mathbf{B}_z \right) \text{,} \\
	_mT_z &\approx \left( {}_m\bm{\sigma} \odot {}_m\mathbf{a} \right)^\mathsf{T} \left( {}_m\mathbf{r}_x \odot {}_m\mathbf{B}_y - {}_m\mathbf{r}_y \odot {}_m\mathbf{B}_x \right) \text{.}
\end{align}

Although these force and torque equations appear complicated, they consist of element-wise multiplication, vector addition, and matrix multiplication. These are trivial for a personal computer to calculate, and as such the equations may be implemented in code resulting in fast force and torque calculations.