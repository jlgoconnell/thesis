
Geometry-based analytic modelling of the magnetostatic fields began in the late twentieth century, and despite decades of research, limitations in current models are present to this day. This thesis attempts to address some of these limitations, with a particular focus on the geometry and permeability of a permanent magnet. Throughout this project, several milestones were achieved, but there is still more scope for future projects on the modelling of permanent magnets. This chapter concludes this thesis by summarising the work completed to date and outlining potential future work following this research.

\section{Primary research outcomes}
The overarching outcome of this thesis is the development of accurate and fast methodologies for the modelling of permanent magnets, and is summarised in this section.

\subsection{Magnetic field equations for ideal polyhedral magnets}
The first major outcome of this project was the development of the magnetic field equations seen in Chapters \ref{chap:paper1} and \ref{chap:paper2}. These equations describe the magnetic field produced by any permanent magnet with polyhedral geometry, constant magnetisation, and a relative permeability of unity. These equations are derived with emphasis on computational efficiency, with each methodology having its own advantages and disadvantages. The first methodology (Chapter \ref{chap:paper1}) is most effective when the field is to be calculated at few points in space, such as when computing the electromagnetic force on a moving charged particle. In contrast, the second methodology (Chapter \ref{chap:paper2}) is most effective when calculating the field at many points, such as when finding the magnetostatic force between several permanent magnets. The equations present in both methodologies are readily programmable in any language on a personal computer. They consist only of logarithmic, trigonometric, and square root terms, and are thus highly computationally efficient, with no special functions or numeric integrals required to evaluate the magnetic field.

Due to their fast computation speed, these equations may be used for parametric optimisation studies where geometry or topology is varied. This can be used on large arrays of magnets while varying the geometry of each magnet to achieve a desired field pattern, such as that shown in Chapter \ref{chap:paper3}. In addition, the computation speed of the field equations allows multiple parameters to be simultaneously varied, such as geometry, magnet arrangement, and magnetisation state.

Furthermore, this methodology may be applied to any magnet composed of flat faces. A magnet with curved faces may be approximated with a polyhedral geometry, and this methodology applied, giving a highly accurate solution to the field produced by any permanent magnet geometry.

\subsection{Modelling magnetic permeability in polyhedral permanent magnets}
While it was shown in Chapters \ref{chap:paper1} and \ref{chap:paper2} that the magnetic field produced by a polyhedral magnet may be quickly evaluated, these methodologies do not consider non-unity relative permeability. Permanent magnets generally have relative permeabilities close to unity, but are usually between 1.05 and 1.15, leading to a modelling error when evaluating these fields, especially in the presence of other field sources. The second major outcome of this thesis was the development of a methodology to model magnetic materials with non-unity but constant relative permeability, described in Chapter \ref{chap:paper4}. Again, this methodology was developed with emphasis on computational efficiency, and attempts to circumvent the drawbacks associated with the methodologies found in literature. While most methods currently available are based on an iterative approach, requiring recalculating the magnetic field several times, this new method involves only a single field evaluation combined with a matrix inversion to solve the system in a single step. This method, in conjunction with the field calculation methodology from Chapter \ref{chap:paper2}, creates a highly effective and fast solution for the magnetic field produced by permeable polyhedral permanent magnets.

This methodology is based on the field calculation technique detailed in Chapter \ref{chap:paper2}, and thus may be applied to any magnet geometry. If magnets in a system have non-polyhedral geometry, they may be approximated with a polyhedral shape, allowing fast and accurate estimation of the fields produced by any permanent magnet geometry with constant non-unity relative permeability.

\subsection{Forces and torques between permeable polyhedral magnets}
Although the magnetic fields produced by polyhedral magnets are of interest, one of the most prominent applications of permanent magnets is in electromechanical actuators, where knowledge of the forces and torques is crucial. Thus, the final outcome of this thesis is the development of an accurate force and torque evaluation methodology for polyhedral magnets. This methodology may be applied to any polyhedral magnet geometry with constant non-unity permeability, as described in Chapter \ref{chap:paper4}, giving fast and highly accurate solutions for these parameters. In addition, non-polyhedral geometries may be approximated, allowing the use of this methodology. This outcome was shown to be significantly faster and more accurate than the industry standard finite element analysis in Chapter \ref{chap:paper4} with the assumptions of constant remanence magnetisation and permeability. This methodology may also be applied in a system with other magnetic field sources, as shown in Section \ref{sec:p4externalFields}, enabling fast and accurate computations of the fields, forces, and torques of any system composed of permanent magnets and external field sources.

\section{Future work}
Though this thesis has outlined methodologies to model permanent magnets of arbitrary geometry and non-unity constant relative permeability, there are still several open questions in this area. This section will outline current gaps in knowledge manifested as potential future work following this thesis.

\subsection{Analytic force and torque}
While the magnetic field equations were found analytically in this thesis by solving a double integral across triangular and trapezial surfaces, the forces and torques were evaluated numerically. In general, the forces and torques may be described by the magnetic charge model, which requires a further double or triple integral of the magnetic field expressions over a polygonal surface or polyhedral volume. Due to the rotation matrices involved, and the highly complicated expressions, these force and torque integrals are extremely difficult to solve analytically. Further work in this area would suggest attempting to find a solution to these integrals based on the field equations presented in this thesis. This would likely be performed again using triangular or trapezial surfaces, but due to the general rotations involved with these systems, these integrals may not have a solution. In this case, the suggested future work would consist of finding accurate approximations to the field distribution functions which are analytically integrable. This may consist of Taylor series representations, Fourier series approximations, or other field estimations. Since magnetic fields are generally well-behaved, with the \(\mathbf{B}\)-field being divergence-free, an accurate estimation in terms of integrable functions is likely attainable. Upon solving these approximate integrals, a set of force and torque equations may be produced, which would be extremely quick to evaluate and highly accurate. However, due to the nature of permeability, it is likely these equations would only be valid for magnetic materials with a relative permeability of unity.

\subsection{Non-linear permeability}
In Chapter \ref{chap:paper4}, a framework was defined in which the magnetic fields, forces, and torques may be evaluated quickly and accurately. However, this assumes a constant permeability in each magnetic material. Future work in this area would be based on deriving similar methodologies but generalising further to include non-linear permeability.

This methodology would likely require a volume mesh rather than a surface mesh, since under non-linear permeability, the assumption that \(\nabla \cdot \mathbf{M} = 0\) is not generally true. If a fine enough volume mesh is applied to a magnetic specimen such that the magnetisation vector field across each volume element is approximately constant, then \(\nabla \cdot \mathbf{M} \approx 0\) inside each element. Thus, the volume integral of the magnetic charge model may be neglected, and only the integrals across the surfaces of each volume element remain. The field equations from Chapters \ref{chap:paper1} or \ref{chap:paper2} could be used to calculate the field from each volume element on each other element, and the total field at each evaluated. Based on the field, the magnetisation vector of each element may be estimated based on the \(BH\) curve of the material. This process may be repeated until the magnetisation vectors converge, allowing the computation of the fields, forces, and torques.

While this methodology is not iteration-free, it allows evaluation of the magnetic fields due to non-linear permeability. In addition, the forces and torques may be easily calculated once the magnetisation vectors converge, similar to the methods found in Chapter \ref{chap:paper4}.

\subsection{Magnetic geometry and topology optimisation framework}
As seen in Chapter \ref{chap:paper3}, the speed of field calculations presented in this thesis allows fast parametric optimisation studies. This may be extended to include the effect of permeability, since magnets in an array often have a strong demagnetising effect on one another. Future work could involve developing a generalised framework for optimisation of magnetic systems, allowing variation of magnet geometry, magnetisation strength, and magnetisation direction. For example, to reduce force ripple in an iron-less electric motor, a particular field pattern based on the coil geometry is desired. An optimisation framework based on the methods found in this thesis could be defined, with magnet sizes and magnetisation strengths varied until a desired field pattern is achieved. Due to the emphasis on computational efficiency of the methods presented in this thesis, optimisation routines could be completed quickly, even under the effects of permeability, allowing fast optimisation of many magnetic systems.

\section{Concluding remarks}
This thesis has presented a set of algorithms that allow evaluation of magnetic fields, forces, and torques due to general permeable polyhedral permanent magnets with emphasis placed on computation speed. This emphasis permits fast optimisation of magnetic systems, allowing variations in magnet geometry, topology, and magnetisation. These methodologies represent a considerable contribution to the field of computational magnetostatics and magnetic modelling, with a focus on generalising and relaxing the strong geometry and permeability constraints currently found in literature. The author hopes that this inspires further research into generalised magnetic modelling and assists designers in further optimising their electromagnetic designs, leading to a future which is as attractive as a magnet on iron.