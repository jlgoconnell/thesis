This paper has examined the differences in the magnetic field produced by frustum permanent magnets and cuboidal permanent magnets. This was done by applying magnetic field equations currently available in literature \cite{OConnell2020a} to a six-faced frustum geometry. Two magnetic systems were considered, allowing a direct comparison between cuboidal magnets and six-faced frustum magnets.

The first magnetic system consisted of a single magnet, with the field being computed at a point directly above the centre of the magnet. In this case, it was shown that an optimised frustum magnet can produce a field stronger than that of an optimised cuboid magnet. However, this increase in field strength is only significant when the field point is close to the magnet. As the field point moves further from the magnet, this increase becomes negligible.

The second magnetic system was a two-dimensional Halbach array consisting of tessellated frustum magnets. For this configuration, the amplitude of the field and how closely the field represents a two-dimensional sinusoid defined a cost function \(C\) (to be maximised), which was used to optimise the system. It was found that most optimal frustum geometries were close to cuboidal. As a comparison, the optimal cuboidal arrays were found, showing that in most cases, the optimal frustum topology has negligible effect on the cost function \(C\). Under certain conditions, the optimal frustum array leads to a significant increase in \(C\) over the optimal cuboid array. However, these conditions lead to a relatively weak field, and other measures should be taken to increase \(C\) such as increasing magnet volume or decreasing the array pole pitch.

This paper shows that although frustum magnets can produce more desirable magnetic fields than cuboidal magnets, this effect is insignificant in many cases. Additionally, complicated magnet geometries such as frusta are more expensive to produce than simple geometries such as cuboids. Hence, the cost associated with the manufacture of these magnets is likely not worth the advantage of a more desirable field. Measures such as varying magnet volume and size are likely to be more effective than using complicated magnet geometries. Furthermore, for multi-magnet arrays, optimising magnet topology and magnetisations is likely more effective than varying the geometry of individual magnets.