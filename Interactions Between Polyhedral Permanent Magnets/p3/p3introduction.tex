Cuboidal permanent magnets have been studied extensively in literature, with the first three-dimensional field equations derived in 1984 by \textcite{Akoun1984}. Since then, more general equations have been derived describing the fields produced by cuboidal magnets. \textcite{Bancel1999} showed that a cuboidal magnet is equivalent to a set of magnetic point charges located at the vertices when calculating the field, and \textcite{Ravaud2009} presented field equations for a cuboidal magnet with an arbitrary magnetisation direction. Other studies have explored the forces between two parallel cuboidal magnets \cite{Akoun1984,Janssen2009a}, the torque between two parallel cuboidal magnets \cite{Allag2009,Janssen2011,Janssen2010}, and the force between cuboidal magnets under rotations about a single axis \cite{Dam2016}.

The geometry of a permanent magnet has a considerable effect on the field it produces, and as such, modifying the geometry of a six-sided magnet from a conventional cuboid may lead to a more desirable field. By angling two pairs of opposite faces of a cuboid, a six-sided frustum is created, such as the one shown in Figure \ref{fig:p3myFrustum}. Due to the angled faces being trapezial instead of rectangular, the field equations become significantly more difficult to derive, and thus frustum magnets have been seldom studied. However, several approaches have been used to derive field equations produced by polyhedral magnets \cite{Compter2010,Fabbri2008,Janssen2010a,Janssen2009,OConnell2020,Rubeck2013,OConnell2020a}, which can be applied to frustum magnets.

In addition to single magnets, considerable research has been conducted on magnet arrays. In particular, the linear Halbach array, first theorised by \textcite{Mallinson1973}, has been studied extensively. In recent years, this has lead to further investigation into planar Halbach arrays for use in planar actuators, for which studies have modelled the magnetic field using cuboidal magnets \cite{Boeij2006,Rovers2012} and triangular prismatic magnets \cite{Cho2001} with magnetisations parallel to a principal axis. Further studies have been conducted to optimise the magnetic field produced by planar Halbach arrays using cuboidal magnets \cite{Huang2008,Jansen2008} and triangular or trapezoidal prismatic magnets \cite{Cho2002,Peng2013}. \textcite{Min2010} has considered cuboidal magnets with magnetisations which are not parallel to a principal axis, leading to a field with a stronger \(z\)-component with smaller high-order harmonics. However, all aforementioned studies are limited to magnets with faces either parallel or orthogonal to the \(z\)-axis; i.e. the normal vector of every magnet face is parallel to the \(z\) axis or lies in the \(XY\)-plane. There exists little-to-no research on magnets with faces rotated about the \(x\)- or \(y\)-axes.

This paper applies previous polyhedral magnet field equations to frustum magnets in order to explore potential advantages they have over cuboidal magnets. In Section \ref{sec:p3frustumField}, the method used to calculate the magnetic field produced by a frustum magnet is outlined. This methodology is applied to a square-based right frustum permanent magnet in Section \ref{sec:p3pyramidFrustum}, where the geometry is varied to optimise the magnetic field strength at a point above the magnet. Section \ref{sec:p3planarArray} presents a two-dimensional permanent magnet Halbach array, where the magnetic field is optimised by varying the magnet geometry. These geometries are compared to the equivalent cuboidal geometries, and any advantage in frustum magnets quantified.