Magnetism has captivated the imagination of humanity for thousands of years and has resulted in some of the greatest innovations humankind has ever seen. It is a concept so simple at heart, but becomes incredibly complicated in the details; a child can comprehend the basic ideas, but years of dedication must be given to understand what magnetism truly is. Given two permanent magnets, anyone can quickly come to the conclusion that one side of a magnet attracts one side of another, but repels the other side, and vice versa. However, understanding, quantifying, and elucidating that attraction or repulsion has presented a monumental challenge to scientists and engineers. This short chapter gives a brief overview of the development of our current understanding of magnetism, dating back to the discovery of the lodestone, to our current understanding of electron angular momentum.

\section*{The lodestone}
The first observations of magnetism came with the discovery of the lodestone, a rare naturally occurring magnet likely magnetised by lightning strikes \cite{Blundell2012}, but exactly when and where this discovery took place is still debated to this day. Legend says that the shepherd Magnes, while pasturing his flocks, found his iron-nailed shoes and iron-tipped cane attracted to a strange stone in the ground around 1000 BC \cite{Mourino1991,Stohr2006,Hafeli1997,Stoner1934}. However, Magnes was probably a mythical figure \cite{Mourino1991}, and the discovery of the lodestone is often attributed to the Greeks around 600 BC based on writings by Thales \cite{Mills2004,Stohr2006,Stoner1934,Morrish1965}, but may have been as early as the 26th to the 28th century BC by the Chinese \cite{Hafeli1997,Mattis1981,Stoner1934,Morrish1965}. While it is unclear whether they had discovered the lodestone before the Greeks, the Chinese were using the stone for geomancy by the third or fourth century BC, believing that good fortune could be achieved by aligning housing, beds, and other structures with the heavens \cite{Hafeli1997,Stohr2006}. During the fourth century BC, the Chinese began using the lodestone for navigation purposes, but this was limited to land-based navigation \cite{Hafeli1997}; the magnetic compass would not be used for sea-based navigation for a further thousand years.

By the end of the 11th century AD, the Chinese had begun to use the magnetic compass to navigate the waves \cite{Mills2004,Stohr2006,Coey2010,Lacheisserie2005}. Shortly after, the technology found its way to Europe, likely due to Arabic traders, with the Englishman Alexander Neckam writing of it in 1187 AD \cite{Mills2004,Hafeli1997,Coey2010,Stoner1934}. The fascinating device likely inspired the Frenchman Petrus Peregrinus de Maricourt to conduct experiments on magnetism, before writing his treatise describing the properties of magnets and magnetic forces in 1269 AD \cite{Mills2004,Stohr2006,Hafeli1997,Mattis1981,Coey2010,Stoner1934,Morrish1965,Lacheisserie2005}. Through the next few hundred years, little progress was made on magnetism, but the magnetic compass almost certainly accelerated the European discovery of the Americas and aided in building ocean-based trade routes.

\section*{The Renaissance and scientific revolution}
After hundreds of years mastering the magnetic compass, the Europeans had made several observations on magnetism, which would spark new interest in the science. By about 1450, sundial makers had noticed that magnetic north did not align perfectly with true north \cite{Sander2017,Chapman1943}, a phenomenon known as magnetic declination. Shortly after, seafarers noticed this deviation between magnetic north and true north varied with geographic location, and some hoped to calculate location based on the deviation between magnetic north and true north \cite{Sander2017,Hellmann1899}. The value of this deviation was first measured in Rome in 1510 by Georg Hartmann, who described it in a letter to Duke Albrech of Prussia in 1544 \cite{Harradon1943,Chapman1943}. In his letter, Hartmann also described how a magnetic needle does not align with the horizon when suspended on a pivot. Rather, it dips vertically downward \cite{Chapman1943,Harradon1943,Stoner1934} and is known as magnetic inclination. Although this was the earliest documented writing of magnetic inclination, it was never published and lay unnoticed in archive \cite{Harradon1943}. Instead, Robert Norman is often credited for the discovery of magnetic inclination, with his first measurement of the phenomenon in 1576 \cite{Mills2004,Harradon1943,Stoner1934}. These discoveries brought about new interest in the science of magnetism and geomagnetism, leading to more discoveries in the following years, including one of the most influential books published in magnetism at the beginning of the 17th century.

In 1600, Englishman William Gilbert published the groundbreaking \textit{De~Magnete}, outlining observations he had made over the past years \cite{Stohr2006,Mills2004,Hafeli1997,Mattis1981,Coey2010,Stoner1934,Morrish1965,Lacheisserie2005}. In this book, he defined the poles of a magnet, noticing that they cannot be separated; cutting a magnet in two leaves two magnets, each with their own pair of poles. He also noted that a magnet heated beyond a certain temperature lost its magnetisation altogether, effectively observing the Curie temperature in action \cite{Mills2004}. He described how to increase the strength of a lodestone by `arming' it with iron at each pole \cite{Mills2004,Hafeli1997}, which was improved in 1613 by Ridley and further in 1616 by Barlowe \cite{Mills2004}. Potentially the most interesting finding of the book, however, was that Earth behaved like a giant magnet \cite{Mills2004,Stohr2006,Mattis1981,Coey2010,Stoner1934}. Following this, progress in magnetism slowed, with many scientists of the time exploring electricity, not realising the intertwinement between the two sciences.

\section*{The marriage of electricity and magnetism}
A groundbreaking discovery was made in about 1820 when \O{}rsted placed a magnetic compass near a current carrying wire. He noticed the compass needle align itself perpendicular to the wire, implying electricity and magnetism were somehow linked \cite{Stohr2006,Hafeli1997,Coey2010,Selvan2007,Stoner1934,Lacheisserie2005}. This experiment resulted in the development of the new science of electromagnetism and acted as a precursor to some of the most important scientific breakthroughs in history.

Several weeks after \O{}rsted's experiment, Amp\'ere calculated the magnetic force between two current carrying conductors \cite{Stohr2006,Hafeli1997,Coey2010,Selvan2007,Stoner1934}, theorising that the magnetism in materials could be caused by small current loops on the molecular scale \cite{Lacheisserie2005}. That same year, Biot and Savart formulated the Biot-Savart law, describing the magnetic field produced by a current carrying wire which \O{}rsted had observed \cite{Stohr2006,Selvan2007,Stoner1934}. The following year, Faraday discovered electromagnetic induction \cite{Coey2010,Lacheisserie2005}, where the magnetic field produced by one wire can induce a current in another. Several decades later, a particularly enlightening discovery was made by Faraday in 1845, when he found a link between light and magnetism \cite{Mattis1981,Mayer2021,Stoner1934,Knudsen1976}. These discoveries inspired James Clerk Maxwell to publish the famous set of equations which bear his name in 1864 \cite{Selvan2007,Mayer2021,Lacheisserie2005,Maxwell1865}, revolutionising electromagnetic science. Maxwell used his equations to measure the speed of electromagnetic waves, noticing that it matches the speed of light very closely \cite{Mayer2021}, and theorised that light is simply an electromagnetic wave. Toward the end of the century in 1888, Hertz was able to experimentally prove the existence of electromagnetic waves \cite{Mayer2021} by producing radio waves. Interestingly, he believed this discovery insignificant, not foreseeing the worldwide adoption of electromagnetic waves less than a century later.

\section*{Cathode rays and the electron}
The discharge of electricity through various gases had been a topic of scientific interest since the early 18th century, but it wasn't until the middle of the 19th that a bright discovery was made. By applying a large potential difference across an anode and cathode in a partial vacuum, a brilliant glow would be produced, the colour of which dictated by the gas and pressure in the partial vacuum. The first person to have observed this phenomenon was likely German instrument maker Heinrich Gei\ss ler in 1855, who manufactured and enhanced cathode ray tubes made from glass \cite{Arabatzis2009}. Several years later, Julius Pl\"ucker and his student Johann Wilhelm Hittorf had experimented on the glow produced by electric discharge in a cathode ray tube. Pl\"ucker had noticed a fluorescent patch on the wall of the tube where the cathode rays had struck, with Hittorf noting an object placed in the path of the rays cast a `shadow' and that the rays may be deflected by a magnetic field \cite{Davis2007}. Over the next decades, this would excite a new area of science and a fierce debate on the composition of these cathode rays.

In the late 19th century, there was much scientific discussion whether cathode rays were waves or streams of particles, with both theories having experimental observations to support their claims. William Crookes in 1879 postulated that since cathode rays can be deflected by a magnetic field, they must be streams of negatively charged particles \cite{Davis2007}. However, Hertz was unable to deflect these rays with an electric field in 1883, implying they may instead be waves \cite{Davis2007,Arabatzis2009}. Later, Hertz also noted that the rays could not pass through thin gold sheets which were impenetrable by atoms. At a time where atoms were thought to be indivisible and the smallest possible particle, he reasoned that cathode rays could not possibly be particles. To complicate matters further, in 1890 Arthur Schuster was able to estimate the charge to mass ratio of cathode rays based on experimental data, implying they were particles \cite{Arabatzis2009}. Furthermore, in 1895 Jean Perrin conducted an experiment, showing that cathode rays carry negative electric charge, reasoning that they must therefore be negatively charged particles.

The debate was finally settled in the 1890s, with experiments conduced by Joseph J Thomson. Unlike Hertz the previous decade, he was able to deflect cathode rays with an electric field. Hertz had placed electrically conductive plates inside a glass tube to create an electric field. However, the gas in the partial vacuum had become ionised, with negative charges being attracted to the positive plate and vice versa. This minimised the potential difference between the plates, leading to negligible electric field in the tube. Thomson had created a better vacuum, thus minimising this effect and detecting a deflection of the rays \cite{Davis2007}. Thomson had also measured the velocity of the rays, showing they were significantly slower than light, before estimating their charge to mass ratio. These findings heavily suggested the rays were composed of negatively charged particles, and Thomson is widely regarded as discovering the electron with these experiments. This discovery would lead to some of the most significant innovations in science and engineering ever seen.

The 19th century provided immense progress in the understanding of electromagnetism. It began with scientists discovering a link between electricity and magnetism, and ended with the experimental discovery of electromagnetic waves and the electron. It brought the idea that light was simply an electromagnetic wave, and followed the equations set out by Maxwell. This century was possibly the most influential and progressive in terms of electromagnetic science, and would lead to the quantum age in the following century.

\section*{The quantum electron}
It may have been Pieter Zeeman who unknowingly first observed quantum effects with his discovery of the Zeeman effect. In 1896, he was experimenting with atomic spectroscopy of sodium atoms under the influence of a magnetic field \cite{Kox1997}. He noticed that without a magnetic field, the spectrum lines were sharp and narrow, but as soon as a magnetic field was applied, the lines widened, becoming several times thicker \cite{Kox1997,Stoner1934,Mattis1981}. A magnetic field had some effect on the spectrum lines, but it was unclear what exactly this effect was. Several months later, this experiment was repeated for cadmium, and it was observed that the line was not getting thicker, but was actually splitting into several lines which were very close together \cite{Kox1997}. At the time, the structure of the atom was unknown, but Zeeman's teacher, Hendrik Lorentz, theorised this effect was related to charged particles in the atom \cite{Mattis1981,Kox1997}. However, this discovery would be one of the first discoveries related to the nature of the electron and quantum mechanics.

With the discovery of the electron and the invention of quantum mechanics, the Zeeman effect was partially explained. If energy is imparted on electrons orbiting a nucleus, they may transition to higher energy states, before transitioning back and releasing the energy as a photon with a particular wavelength. However, under the influence of a magnetic field, the orbital angular momentum of an electron as it rotates about the nucleus creates its own magnetic field, which interacts with the external field. Thus, the energy level depends on the angular momentum, and different values of angular momentum lead to small changes in the photon wavelength. Earlier, Neils Bohr had predicted that the orbital angular momentum of an electron had to be quantised, which explained why the spectral lines were splitting rather than widening. While this explained much of the Zeeman effect, there were some discrepancies remaining \cite{Mattis1981}, which would require the concept of electron spin.

\section*{Spin}
To explain the discrepancies in the Zeeman effect, it was theorised that electrons not only rotate around a nucleus, but also rotate on their own axis, therefore creating an additional magnetic field component. Although this was later found to be false, and this spin was an inherent quantum property of the electron, the term \textit{electron spin} stuck and is still used today. To explore electron spin, Otto Stern proposed an experiment in 1921, which was successfully conducted by Walther Gerlach in 1922 \cite{Mattis1981}. In this experiment, silver atoms were fired through a spatially varying magnetic field before colliding with a detector. Each silver atom was deflected a small amount due to the electron spin interacting with the magnetic field. If the spin was not quantised, the detector would show a continuous distribution of collisions. However, rather than a distribution, two discrete collision sites were observed, implying that electron spin must be quantised, and could attain one of two values: \(+1/2\) or \(-1/2\) \cite{Morrish1965}.

Electron spin forms much of our current understanding of magnetism. For example, paramagnetic materials such as aluminium contain an unpaired electron in the outer electron shell which has a spin magnetic moment and thus interacts with magnetic fields. This electron tends to align its spin moment with an applied magnetic field, and is therefore weakly attracted to the source of the field. In contrast, ferromagnetic materials such as iron exhibit a phenomenon called exchange interaction, where the exchange energy is minimised if the unpaired electrons of nearby atoms have identical spins~\cite{Stoner1934}. Since nearby unpaired electrons have the same spin, their magnetic moments combine to create a far stronger net magnetic moment, allowing strong magnetic interactions.

Although much of the journey to discover electron spin was somewhat unrelated to magnetism, it has allowed us to understand how magnetic materials behave. Without the monumental discoveries of quantum physicists in the early 20th century, our understanding of magnetism would be primitive compared to what it is today.

\section*{The development of permanent magnets}
Since their inception, significant developments in permanent magnet materials has enabled stronger magnetisations, greater resistance to demagnetisation, corrosion resistance, and more. This began when scientists in the 18th century began manufacturing permanent magnets. The first permanent magnets to be manufactured were likely produced by Gowin Knight \cite{Moskowitz1995} by grinding iron oxide into fine particles and mixing with water to create a slurry. He would then combine the slurry with linseed oil to create a paste, before moulding, baking, and magnetising with a separate magnet. However, few developments in permanent magnetism would occur throughout the rest of the 18th and 19th centuries.

During the early 20th century, considerable improvements were made with the development of cobalt steels in around 1920 \cite{Zhukov2016}. In the following decade, Alnico magnets were produced, followed by barium ferrites in the 1950s. At this point, permanent magnets were still fairly weak, but this changed with the invention of strong samarium cobalt magnets in the late 1960s. Unfortunately, these stronger magnets were costly, necessitating a new permanent magnet material. This material would come to light in the early 1980s with the introduction of neodymium magnets, which, to this day, dominate the high-performance permanent magnet market. Currently, neodymium permanent magnets are able to achieve a remanence magnetisation of almost 1.5\si{\tesla} and are resistant to demagnetisation. However, they are not perfect and are sensitive to corrosion and temperature.

\section*{Magnetism today}
Our understanding of magnetism began with the observation that iron nails were occasionally attracted to some rocks from the ground and would often align toward the north pole. A long voyage of experimentation and discovery later, we understand the quantum effects of magnetism, and how electrons arranged in particular configurations can drastically affect the magnetic properties of a material. Materials and manufacturing sciences have allowed us to create strong permanent magnets, but these are not perfect. While we currently have some understanding of magnetic materials and the subatomic and quantum effects responsible for magnetism, there are still many open questions in the physics of magnetism. Furthermore, how can we, as engineers, take advantage of these phenomena? Emerging fields such as spintronics aim to exploit the physical science of magnetism to further improve current technology. However, these would not have been possible without the tremendous discoveries of those who have preceded us.

\vspace{1cm}
\begin{quoting}
    If I have seen further it is by standing on the shoulders of Giants.\flushright{---Isaac Newton}
\end{quoting}

\newpage
\printbibliography[title=References]