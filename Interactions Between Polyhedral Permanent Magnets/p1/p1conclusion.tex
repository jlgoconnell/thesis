Permanent magnets are widely used in many industries and as such it is useful to characterise the interactions between them. This paper has outlined a fast semi-analytic method to calculate magnetic fields, forces, and torques of polyhedral permanent magnets. The development of two algorithms were discussed and implemented in Matlab. Several validation cases were considered, including a basic cuboid case and a more complicated system with dodecahedral magnets. These results were then validated against both literature (where possible) and finite element simulations. Then, a system with two pyramidal frustums was implemented in which the wall angle and separation distance could be varied while maintaining constant magnetic volume and height. The algorithms presented in this paper were used for rapid optimisation\footnote{Here, optimisation refers to maximising the force between two magnets of constant volume by varying magnet geometry. However, many forms of optimisation, such as optimisation of force per unit magnet height or torque per unit mass, may be used.} of this system, maximising the force between the magnets at a given distance. This resulted in the optimal angle being calculated over a large number of separations. Additionally, a considerable improvement in force over cuboidal magnets was found, showing standard magnetic geometries are not always optimal. The algorithms presented here can be used to further understand non-standard magnetic geometries, as well as optimise magnetic systems quickly to increase their performance.