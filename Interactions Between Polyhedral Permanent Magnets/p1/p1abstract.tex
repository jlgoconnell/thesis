This paper outlines an algorithm which analytically calculates the magnetic field produced by a general polyhedral permanent magnet with any number of faces and arbitrary face orientations, then uses the algorithm to \mbox{semi-analytically} calculate the force and torque on a second general polyhedral magnet. The algorithm is validated against both literature and finite element simulations using cuboids and dodecahedra. It is then used to model a basic two-magnet repulsive system, where it is shown that frustum magnets can produce a larger force per unit volume than cuboidal magnets. The shape of the frustums is optimised to maximise the force between them at a given separation distance, showing a considerable increase in force when compared to cuboidal magnets with the same volume. This paper shows that there is scope to improve performance of magnetic systems by using novel magnet shapes, and presents an algorithm which can be used for this optimisation process.