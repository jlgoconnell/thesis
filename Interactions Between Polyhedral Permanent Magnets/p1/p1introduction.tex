Permanent magnets have many applications in magnetic resonance imaging, gearing, actuators, and motors \cite{Furlani2001}. Furthermore, they can be an essential component in wave energy harvesting \cite{Mann2010}, vibration isolation \cite{Robertson2009}, and many other applications. Due to their extensive use in a number of industries, it is important to understand the interactions between them, improving the design and optimisation process of electromagnetic systems.

Throughout recent decades, researchers have attempted to understand interactions between permanent magnets. \textcite{Akoun1984} started this trend by calculating the force between two parallel cuboidal permanent magnets with parallel magnetisation by finding the interaction energy between the magnets. \textcite{Janssen2010} used the same energy-based approach, but instead calculated the torque on one of the magnets. \textcite{Allag2009} extended the force and torque expressions to non-parallel magnetisations. \textcite{EngelHerbert2005} and \textcite{Ravaud2009} have calculated the magnetic field of cuboidal magnets rather than forces and torques. However, these studies are limited to cuboidal magnets and cannot be used for any other shapes.

A number of studies have examined ring and cylindrical magnets rather than cuboids. \textcite{Furlani1995} was able to semi-analytically calculate the field due to radially magnetised ring magnet sectors. Several papers by Ravaud et al.\ \cite{Ravaud2008,Ravaud2008a,Ravaud2009a} have found expressions for radially and axially magnetised ring magnets and sectors. Again, however, these studies are limited to ring magnets and cannot be used for other geometries.

To mitigate this geometrical limitation, some researchers have explored polyhedral permanent magnets, generalising the solution to any three dimensional shape with flat facets. Some authors such as \textcite{Soltner2010} and \textcite{Meessen2008} have approximated the magnetic field of a polyhedral magnet using assumptions such as the dipole model or discretisation of shapes into cuboids, but these are not always accurate. Other authors, however, have been more successful in solving for the exact magnetic field. \textcite{Janssen2010a} and \textcite{Rubeck2013} were able to find analytic expressions for the magnetic field of a polyhedral magnet by decomposing it into a collection of simple two-dimensional planar surfaces. \textcite{Meessen2008} and \textcite{Lee2004} studied trapezoidal magnets in a Halbach array using discretised magnets and magnets of infinite thickness respectively and found improvement in the maximum magnetic field strength over more traditional cuboidal Halbach arrays. However, there has been little work on the forces and torques due to polyhedral permanent magnets.

Several studies by \textcite{Beleggia2003,Beleggia2005}, \textcite{Beleggia2004}, and \textcite{DeGraef2009} examine magnetic nanoparticles with arbitrary shape. They found expressions for the demagnetisation tensor field, interaction energy, force, and torque using a Fourier space approach. However, for most shapes, these must be calculated numerically, limiting the accuracy and speed of the solution.

This paper outlines a semi-analytic method for calculating the magnetic fields, forces, and torques for polyhedral permanent magnets. First, an algorithm to analytically calculate the magnetic field is presented. This method is similar to the magnetic field calculation method given by \textcite{Rubeck2013} but requires evaluation of fewer terms by using general scalene triangles rather than right-angled triangles. Then, numeric integration is performed to find the force and torque on a second polyhedral magnet due to the field from the first. This work is validated using past literature and finite element simulations on the magnetic configuration presented by \textcite{Akoun1984}. To further validate the algorithm, finite element simulations are performed on two perpendicularly magnetised dodecahedral magnets, which are compared to the semi-analytic solutions from this method. Once validated, a configuration involving two pyramidal frustum magnets is presented, where it is shown that the frustums can produce a larger repulsive force per unit volume than cuboidal magnets. Finally, this algorithm is used to optimise the geometry of the frustums to maximise the force between them.