Permanent magnets are used in a wide variety of applications, from microphones and loudspeakers to energy harvesting devices \cite{Coey2002}. They are an essential component of many electromechanical systems and are seeing widespread use with the current trend toward electric vehicles. In 2014, permanent magnets had an annual market of \$7bn USD, with hundreds of thousands of tons of neodymium magnets being manufactured annually \cite{Coey2014}. Permanent magnets see wide use in society, and it is therefore important to understand and model them effectively.

With the development of stronger magnetic materials in the late twentieth century, mathematical modelling of permanent magnets has become more common. Magnetic modelling has a high dependence on geometry, and as such, most research in this area has been undertaken on simple geometries such as cuboidal and ring-shaped permanent magnets.

Cuboid magnets are extremely prevalent in society and industry, and have therefore received considerable attention in literature. Additionally, their geometry is simple, leading to simple field equations. In 1984, \textcite{Akoun1984} published the magnetic field equations for a vertically-magnetised cuboid magnet by considering a magnetic charge distribution over the surface of the magnet. \textcite{Bancel1999} suggested that the magnetic field produced by a cuboid can be thought of as field contributions from each of the eight vertices, or `nodes'. Several decades later, \textcite{Ravaud2009} extended these equations to include an arbitrary magnetisation direction by using superposition of mutually perpendicular magnetisations. However, these equations do not consider mathematical singularities, which often exist at points inline with a magnet edge. Other authors have gone beyond field calculations by publishing equations describing forces and torques between parallel cuboid magnets with parallel magnetisations \cite{Akoun1984,Allag2009}, parallel cuboid magnets with arbitrary magnetisations \cite{Janssen2011}, and rotated cuboid magnets \cite{Dam2016}. However, all aforementioned equations are limited to cuboid magnets, and are invalid for other magnet geometries.

Like cuboid magnets, cylindrical and ring-shaped magnets have seen wide use in industry, and have also received considerable attention in literature. They exhibit simple geometry in a cylindrical coordinate system, leading to relatively simple field equations. In 1995, \textcite{Furlani1995} derived the magnetic field due to radially-magnetised ring sectors, but these equations are not fully analytic and require some numerical integration. Ravaud et al.\ \cite{Ravaud2008a,Ravaud2010} improved these equations by using elliptic integrals, leading to fewer numeric integrals. In another study, \textcite{Ravaud2008} considered the simpler geometry of a full ring magnet rather than a sector. With this assumption, they found expressions for the magnetic field produced by both radially- and axially-magnetised ring magnets using elliptic integrals. More recently, \textcite{Caciagli2018} derived simpler equations describing the magnetic field produced by cylindrical magnets. However, these equations are again limited to a specific geometry and other geometries require separate solutions.

Although cuboidal and ring magnets have been studied extensively, few other geometries have received attention. Papers by \textcite{Janssen2009,Janssen2010a}, \textcite{Compter2010}, \textcite{Rubeck2013}, and \textcite{OConnell2020} presented analytic equations for the magnetic field due to a general polyhedral permanent magnet with constant uniform magnetisation and unity relative permeability. These equations can be used for any magnet composed of flat faces, and can also be used to approximate curved surfaces, allowing approximate solutions of any magnet geometry. However, the studies by \textcite{Janssen2009,Janssen2010a} and \textcite{Compter2010} present equations which are not fully simplified, limiting their computational efficiency. The equations presented by \textcite{Rubeck2013} and \textcite{OConnell2020} are simpler, but inefficient for a large number of field evaluations because they require processing of the geometry for every field point. Furthermore, these studies do not consider mathematical singularities, which cause the magnetic field evaluations to become undefined at some locations.

This paper attempts to alleviate both of these issues by presenting new magnetic field equations for a general polyhedral permanent magnet which are fully simplified, computationally efficient, and include singularity treatment. These field equations are derived by the authors in Section \ref{sec:p2methodology} and validated using both past literature and finite element simulations in Section \ref{sec:p2validation} before the paper is concluded in Section \ref{sec:p2conclusion}.