This paper has outlined a new fully analytic and fast method for calculating the magnetic field produced by a polyhedral permanent magnet. The methodology was presented, outlining the process of decomposition and giving the solution to the charge model for a charged trapezium, leading to the total field produced by the polyhedral magnet. Singular regions were identified, and their solutions presented. This was followed by a validation section showing strong agreement between finite element simulations, literature, and the work detailed here. The magnetic field equations presented here are less complicated and more efficient than those in current literature since only two rotations are required per facet, independent of the number of magnetic field calculations.

The equations had high accuracy when compared to finite element simulations and literature using both a pyramid frustum magnet and a cylindrical magnet. The equations were validated against the finite element simulations and literature, giving a maximum error of less than 1 percent for both the frustum and cylindrical magnets.

It was shown that curved surfaces can be approximated as polyhedral surfaces and this methodology applied. A cylindrical magnet was approximated as a polygonal prism with an arbitrary number of sides. It was shown that both the accuracy and computation time increased with the number of sides. For the case considered, it was shown that the error was less than 0.1 percent using a 32-gon prismatic magnet, taking less than 30\si{\milli\second} to compute the field.

Currently, finite element simulations are usually used to analyse complicated magnetic systems. However, this is slow and optimisation of magnet shape or topology can take a considerable amount of time. The methodology presented here can be used to analyse these complicated magnetic systems far more quickly than finite element simulations. Furthermore, due to the simplicity and speed of calculation, it can be used for approximation of complicated magnet shapes, real-time simulations, or optimisation of magnet shape and topology to quickly arrive at a desired magnetic field.