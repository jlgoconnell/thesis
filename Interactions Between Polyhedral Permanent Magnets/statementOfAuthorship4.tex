\subsection*{Statement of authorship}
\renewcommand{\arraystretch}{1.5}
\begin{tabular}{m{0.25\textwidth} m{0.67\textwidth}}
    \hline \hline Paper title & A non-iterative method to solve for magnetic fields, forces, and torques due to permanent magnets with non-unity relative permeability \\ \hline
    Publication status & Submitted \\ \hline
    Publication details & Submitted to \textit{Journal of Magnetism and Magnetic Materials}, Dec. 2021 \\ \hline \hline
\end{tabular}

\vfill

\subsection*{Principal author}
\begin{tabular}{p{0.25\textwidth} m{0.67\textwidth}}
    \hline \hline Name & James O'Connell \\ \hline
    Contribution & \begin{itemize}
        \setlength\itemsep{-2mm}\small
        \item[-] Idea conceptualisation
        \item[-] Review of relevant literature
        \item[-] Deriving a matrix equation for the surface charge density of a magnet based on the field present and the remanence magnetisation of the magnet
        \item[-] Incorporating Gauss' Law for Magnetism into the system
        \item[-] Applying a constrained least squares methodology to solve an overdetermined matrix equation with constraints
        \item[-] Adapting the system of equations to include an arbitrary number of magnetic bodies and arbitrary external magnetic fields
        \item[-] Applied the surface charge density solution to find the magnetic field at any point
        \item[-] Applied the surface charge density solution to estimate the force and torque imparted on all magnets in the system
        \item[-] Implemented the surface charge, magnetic field, force, and torque methodologies in MATLAB code
        \item[-] Created finite element simulations to verify the algorithms and MATLAB code
        \item[-] Analysed the accuracy of the methodology based on mesh density
        \item[-] Wrote manuscript draft and created all figures
        \item[-] Finalisation of article
        \item[-] Preparation and submission for publication, including author correspondence
    \end{itemize} \\ \hline
    Percentage & 90\% \\ \hline
    Certification & \small This paper reports on original research conducted by the author during the period of Higher Degree by Research candidature and is not subject to any obligations or contractual agreements with a third party that would constrain its inclusion in this thesis. The author listed above is the primary author of this paper. \\ \hline
    Signature & \begin{tabular}{m{45mm} m{10mm} m{20mm}}
    \vspace{0.5mm}\includegraphics[width=0.3\textwidth]{jamesSignature.PNG} & Date: & 8 Dec 2021
    \end{tabular}
\end{tabular}

\vfill
    
\subsection*{Co-author contributions}
By signing this statement of authorship, each author certifies that:
\begin{enumerate}
    \item the candidate's stated contribution to the publication is accurate (as detailed above);
    \item permission is granted for the candidate to include the publication in the thesis; and
    \item the sum of all co-author contributions is equal to 100\% less the candidate's stated contribution.
\end{enumerate}
\begin{tabular}{m{0.25\textwidth} m{0.67\textwidth}}
    \hline \hline Name & Will Robertson \\ \hline
    Contribution & 5\% \\ \hline
    Signature & \vspace{2mm}\includegraphics[height=10mm]{willSig} \\  \hline
    Date & 16 Dec 2021 \\
    \hline \hline Name & Ben Cazzolato \\ \hline
    Contribution & 5\% \\ \hline
    Signature & \vspace{2mm} \includegraphics[height=10mm]{benSig} \\ \hline
    Date & 16 Dec 2021 \\
    \hline \hline \vfill
\end{tabular}
\renewcommand{\arraystretch}{1}
\newpage